
\section{Introduction}

Formal dg-algebras, being algebras that are quasi-isomorphic to their cohomology 
algebra, was introduced in \cite{DGMS} to solve problems in rational homotopy theory. In 
the aforementioned paper, the authors remark that having no no-vanishing Massey 
$n$-products is a weaker property than being formal, meaning that Massey $n$-products 
serve as obstructions to formality. They claim that formality is equivalent to a 
``uniform vanishing'' – a stronger version of just being vanishing. In later times 
the study of dg-algebras have been explored further using the theory of 
$A_\infty$-algebras. Any dg-algebra $A$ can be viewed as a ``trivial'' 
$A_\infty$-algebra, and the general ones can be thought of as homotopy theoretic 
versions of dg-algebras. In \cite{kadeishvili} Kadeishvili proved that the cohomology 
algebra $H(A)$ of a dg-algebra $A$ naturally admits an $A_\infty$-structure in such 
a way that there is a quasi-isomorphism of $A_\infty$-algebras $H(A)\longrightarrow A$. 
The higher products on this $A_\infty$-structure are often claimed to be the Massey 
products, but this is not always true as the $n$'th product on $H(A)$ is not always 
a representative of the Massey $n$-product \cite{detection}. It is however the case that 
the vanishing of these higher products on $H(A)$ is stronger than the vanishing of 
the Massey $n$-products \cite{AHO} – in fact, if all the higher products vanish, then 
the dg-algebra is formal. Hence having vanishing $A_\infty$-structure on $H(A)$ is 
equivalent to $A$ being formal. Using the equivalent definition of formality from 
\cite{keller} this is true almost by definition. 

In \cite{detection} the authors prove that even though the higher products on $H(A)$ might 
not be the Massey products, they are so up to a sum of lower degree products. We use 
this to prove that vanishing Massey products is equivalent to formality in the case 
that the cohomology algebra has trivial products. Hence the vanishing Massey products 
on spaces with vanishing cup products are always uniformly vanishing in the sense of 
\cite{DGMS}. 

There is a property of a space that allows us to know an upper bound for its 
cup-length, namely the Lusternik-Schnirelmann category of the space. This is an 
integer describing how a space can be glued together by cones. If we limit our study 
to spaces with Lusternik-Schnirelmann $1$, we are certain that our cup-length is $0$, 
meaning that we have vanishing products on reduced cohomology. We introduce reduced 
formality and use the above result to show that spaces with Lusternik-Schnirelmann 
category $1$ are reduced formal, and afterward show that the word ``reduced'' is 
redundant.
