
\section{Introduction}

The notion of formal dg-algebras, being algebras that are quasi-isomorphic to their cohomology 
algebra, was introduced in \cite{DGMS} to solve problems in rational homotopy theory. In 
this paper, the authors remark that having no non-vanishing Massey 
$n$-products is a weaker property than being formal, meaning that Massey $n$-products 
serve as obstructions to formality. They claim that formality is equivalent to a 
``uniform vanishing'' – a stronger version of just being vanishing. In later times 
the study of dg-algebras have been explored further using the theory of 
$A_\infty$-algebras. Any dg-algebra $A$ can be viewed as a ``trivial'' 
$A_\infty$-algebra, which ables us to think about general $A_\infty$-algebras as homotopy 
theoretic versions of dg-algebras. In \cite{kadeishvili} Kadeishvili proved that the 
cohomology algebra $H(A)$ of a dg-algebra $A$ naturally admits an $A_\infty$-structure in 
such a way that there is a quasi-isomorphism of $A_\infty$-algebras $H(A)\longrightarrow A$. 
The higher products on this $A_\infty$-structure are often claimed to be the Massey 
products, but this is not always true as the $n$-ary product of the $A_\infty$-structure on 
$H(A)$ is not always a representative of the Massey $n$-product \cite{detection}. It is 
however the case that the vanishing of these higher arity products on $H(A)$ is a stronger 
condition than the vanishing of the Massey $n$-products \cite{AHO} – in fact, if all the 
higher products vanish, then the dg-algebra is formal. Hence having vanishing 
$A_\infty$-structure on $H(A)$ is equivalent to $A$ being formal. Using the equivalent 
definition of formality from \cite{keller} this is true almost by definition. 

In \cite{detection} the authors prove that even though the higher products on $H(A)$ might 
not be the Massey products, they are so up to a sum of lower degree products. We use 
this to prove that vanishing Massey products is equivalent to formality in the case 
that the cohomology algebra has trivial products. Hence the vanishing Massey products 
on spaces with vanishing cup products are always uniformly vanishing in the sense of 
\cite{DGMS}. 

There is a property of a space that allows us to know an upper bound for its 
cup-length, namely the Lusternik-Schnirelmann category of the space. This is an 
integer describing how a space can be glued together by cones. If we limit our study 
to spaces with Lusternik-Schnirelmann $1$, we know that the cup-length is $0$, 
meaning that we have vanishing products on reduced cohomology. Using the above result
we then prove that spaces with Lusternik-Schnirelmann category $1$ are formal. This 
generalizes the formality for suspended spaces, proven in \cite{FHT}.

The result is most likely already known by specialists, but this seems to be a new method 
of proving it. Alternatively one can use that any space $X$ with $\text{cat}_{LS}(X)=1$ 
is a co-H-space \cite{hess}, and then that any co-H-space is a wedge of spheres 
\cite{co-H-space}. Formality is preserved under the wedge product \cite{hess}, and since 
spheres are formal, we know that any co-H-space, and thus any space $X$ with 
$\text{cat}_{LS}(X)=1$ is a formal space. 

\begin{acknowledgements}
This work is the product of the authors master thesis at NTNU
supervised by Gereon Quick. We want to thank him for an interesting project, and for 
all the helpful feedback and comments. We also thank José Manuel Moreno-Fernández for 
providing comments on an earlier version of this paper. 
\end{acknowledgements}