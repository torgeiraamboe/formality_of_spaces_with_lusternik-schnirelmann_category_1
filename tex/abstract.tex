\begin{abstract}
    There is a property of a space that allows us to know an upper bound for its 
    cup-length, namely the Lusternik-Schnirelmann category of the space. This is 
    an integer describing how a space can be glued together by cones. If we limit 
    our study to spaces with Lusternik-Schnirelmann $1$, we are certain that our 
    cup-length is $0$, meaning that we have vanishing products on reduced cohomology. 
    We introduce reduced formality and use the above result to show that spaces 
    with Lusternik-Schnirelmann category $1$ are reduced formal, and afterward show 
    that the word ``reduced'' is redundant.
\end{abstract}