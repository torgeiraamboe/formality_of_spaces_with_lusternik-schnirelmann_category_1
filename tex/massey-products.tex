

\section{Masse products}

The other important part of this story is the Massey products. These are partially defined 
higher order cohomology operations on dg-algebras. Since they are only partially defined 
we need an easy way to package this information in. This is done through defining systems. 
Let $A$ be a dg-algebra, and denote $\bar{x} = (-1)^{|x|}x$. 

\begin{definition}
    A defining system for a set of cohomology classes $x_1, \ldots, x_n$ 
    in $H(A)$ is a collection $\{ a_{i,j}\}$ of cochains in $A$ such that
    \begin{enumerate}
        \item $[a_{i-1, i}] = x_i$
        \item $d(a_{i, j}) = \sum_{i<k<j}\overline{a_{i, k}}a_{k, j}$
    \end{enumerate}
    for all pairs $(i,j)\neq (0,n)$ where $i\leq j$.
\end{definition}

\begin{definition}
    The Massey $n$-product of $n$ cohomology classes $x_1, \ldots, x_n$, 
    denoted $\langle x_1, \ldots, x_n\rangle$ is defined to be the set of all $[a_{0,n}]$, 
    where $$a_{0,n} = \sum_{0<k<n}\overline{a_{0, k}}a_{k, n}$$ such that $\{ a_{i,j} \}$ 
    is a defining system.         
\end{definition}

For $n=2$ this is just the induced product on cohomology, up to a sign. For $n=3$ this 
is the classical triple Massey product. When we use the phrase ``all Massey $n$-products'', 
we mean all Massey $n$-products for $n\geq 3$. 

The fact that multiple cohomology classes can be in this set means that the product is 
only partially defined. If this set contains just a single class, then we say the Massey 
product is uniquely defined. What matters for us is when these products are trivial. 
Since they are partially defined what we mean is the following.

\begin{definition}
    We say that the Massey $n$-product vanishes if it contains zero as an 
    element, i.e. $0\in \langle x_1, \ldots, x_n\rangle$.    
\end{definition}
