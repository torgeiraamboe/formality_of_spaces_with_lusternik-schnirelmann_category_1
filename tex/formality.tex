

\section{Results on formality}

One of the reasons we are interested in Massey products is that they serve as an 
obstruction to formality. Intuitively, if Massey products exist on the algebra of 
cochains $C(X)$ on a topological space $X$, this means that $C(X)$ contains more 
information about our space than the cohomology ring of $X$. If a non-trivial Massey 
product exists on $C(X)$, this means that there will always exist another space $Y$, 
not homeomorphic to $X$, but with the same cohomology ring as $X$. The most famous 
example of this are the Borromean rings. These are a set of three rings, every pair of 
them not linked with the two others, but all three still linked. This has the same 
cohomology ring as three unlinked circles, but the Borromean rings admit a non-vanishing 
Massey product, while the three unlinked circles does not. Hence the algebra of forms 
on the Borromean rings can not be formal. This means that Massey products detect 
non-formality. This is summarized into the following theorem.

\begin{theorem}[\cite{DGMS}]
    Let $(A, m)$ be a formal dg-algebra. Then all Massey $n$-products vanish.         
\end{theorem}

Unfortunately, knowing a dg-algebra has no non-vanishing Massey products is not enough 
to determine that it is formal. But using the full $A_\infty$-algebras we can get closer 
to some version of this being true.

\begin{theorem}[\cite{kadeishvili}]
    \label{thm:Kadeishvili}
    Let $(A,m)$ be a dg-algebra. Then there exists an (up to $A_\infty$-isomorphism) 
    unique $A_\infty$-algebra structure on its cohomology algebra $H(A)$ with $m_1=0$, 
    $m_2$ the induced product from $A$, and a quasi-isomorphism of $A_\infty$-algebras 
    $H(A)\longrightarrow A$.
\end{theorem}

Note that this does not mean that all dg-algebras are formal, as now $H(A)$ is not 
necessarily just a dg-algebra anymore. We can think of this higher structure on $H(A)$ 
as measuring how far away $A$ is from being formal. Since $m_1=0$ we get that the 
product $m_2$ is associative, but not for the reason we mentioned earlier. This means 
that these higher products no longer are interpreted as homotopies, but instead as 
something more like Massey products. Hence we call them the ``higher products'' on $H(A)$.

We said that the $A_\infty$-structure measures how far away $A$ is from being formal. 
We noted earlier that an $A_\infty$-algebra with $m_k=0$ for $k\geq 3$ is a dg-algebra. 
This means that if the $A_\infty$-structure on $H(A)$ has $m_k=0$ for $k\geq 3$, then 
$A$ is formal, as we then have an $A_\infty$-quasi-isomorphism between two dg-algebras. 
This is rather important so we state it as a theorem.

\begin{theorem}[\cite{AHO}]
    \label{thm:AHO_formal}
    Let $(A, m)$ be a dg-algebra. Then $A$ is formal if and only if all the higher products
    on $H(A)$ vanish.        
\end{theorem}

One direction of the proof is by definition, as described above. The other part is 
because the $A_\infty$-structure on $H(A)$ is unique up to isomorphism of 
$A_\infty$-algebras.

In \cite{DGMS} the authors say that formality is equivalent to a uniform vanishing of the 
Massey products. We are intuitively able to choose the zero element as our Massey 
products in such a nice uniform way that they are a part of an $A_\infty$-structure, 
which must mean that we have formality, as above.

We want to use this to get an idea of ``how close'' the normal Massey products are 
from being sufficient obstructions. What we mean by this is that we want to find a case 
where vanishing Massey products mean we have a formal dg-algebra. To get to such a result 
we first need a theorem that connects the higher products to normal Massey products.

\begin{theorem}[\cite{detection}]
    Let $A$ be a dg-algebra and $x\in \langle x_1, \ldots, x_n\rangle$ with $n\geq 3$. 
    Then for any $A_\infty$-structure on $H(A)$ we have 
    $$\epsilon m_n(x_1, \ldots, x_n) = x+\Gamma$$ where 
    $\Gamma \in \sum_{j=1}^{n-1}\text{Im}(m_j)$ and 
    $\epsilon = (-1)^{\sum_{j=1}^{n-1} (n-j)|x_j|}$.        
\end{theorem}

\begin{corollary}[\cite{detection}]
    \label{cor:detection_unique}
    Let $A$ be a dg-algebra and $m$ an $A_\infty$-structure on its cohomology $H(A)$ such 
    that $m_k = 0$ for all $1 \leq k \leq n-1$. Then the Massey $n$-product 
    $\langle x_1, \ldots, x_n \rangle$  is uniquely defined for any set of cohomology 
    classes $x_1, \ldots, x_n$. Furthermore the unique element in the Massey product is 
    recovered by $m_n$ up to a sign, i.e. 
    $\langle x_1, \ldots, x_n \rangle = \epsilon m_n(x_1, \ldots, x_n)$, where again 
    $\epsilon = (-1)^{\sum_{j=1}^{n-1} (n-j)|x_j|}$.
\end{corollary}

By this we get our first result.

\begin{theorem}
    \label{thm:1}
    Let $A$ be a dg-algebra and $H(A)$ its cohomology algebra. If the induced product on 
    $H(A)$ is trivial and all Massey $n$-products on $A$ vanish, then $A$ is formal.        
\end{theorem}

\begin{proof}
    By \cref{thm:Kadeishvili} we know that $H(A)$ can be equipped with the structure 
    $\{m_i\}$ of an $A_\infty$-algebra such that $m_1=0$ and $m_2$ is the product induced 
    from $A$, which is assumed to be trivial. We claim that $m_k = 0$ for all $k\geq 3$ 
    as well, and hence that $A$ is formal by \cref{thm:AHO_formal}. We prove this 
    claim by induction. 
    
    Since $m_2=0$ we know that all Massey triple products are defined. By the below 
    induction argument all the higher Massey products will be defined as well. Since 
    $m_2=0$ we already have our base case.
    
    Assume $m_k = 0$ for $1\leq k\leq n-1$. By \cref{cor:detection_unique} we know 
    that $\langle x_1, \ldots, x_n \rangle$ consists of a unique element for all choices of 
    classes $x_1, \ldots, x_n$. This element is by assumption the zero class, as we assumed 
    all Massey products to be vanishing. This class is recovered up to a sign by $m_n$, 
    which means $m_n(x_1,\ldots, x_n)=0$ for all choices of $x_1, \ldots, x_n$. Hence 
    $m_n=0$ and we are done. 
\end{proof}

This proof shows that when we have a trivial induced product on $H(A)$, the vanishing 
Massey products neatly forms a trivial $A_\infty$-structure on $H(A)$, which we earlier 
said was the way of interpreting the uniform vanishing.

It is tempting to think that having trivial product in cohomology also makes every attempt 
to build and produce a Massey product impossible. But there are examples of this not being 
the case. One example is the free loop space of an even-dimensional sphere. Its cohomology 
algebra has trivial product, and it is shown in \cite{nonformal_loop} to have non-zero 
Massey products. Hence it can not be formal. We also mentioned the Borromean rings earlier, 
which gives another example. 

\begin{question}
Is there a more general procedure for choosing elements in all the Massey 
products in such a way that they form an $A_\infty$-structure?
\end{question}