
\section{\texorpdfstring{$A_\infty$}{A}-algebras}

We provide a quick overview of the most important theory on $A_\infty$-algebras. 
For a more in-depth treatment, see \cite{keller}. 
Let $K$ be a field of characteristic $0$. 

\begin{definition}
    An $A_\infty$-algebra $(A, m)$, over $K$ is a $\Z$-graded vector 
    space $A=\bigoplus_{i\in \Z}A_i$ together with a family of $K$-linear maps 
    $m_n : A^{\otimes n}\longrightarrow A$ of degree $2-n$, such that the identities
    $$\sum_{r+s+t = n}(-1)^{r+st}m_{r+1+t} (Id^{\otimes r}\otimes m_s \otimes Id^{\otimes t}) = 0$$
    hold for all $n, s\geq 1$ and $r, t\geq 0$.
\end{definition}

These relations are called the coherence relations, or the Stasheff identities in $A$. 
For $n=1$ the coherence relation simply becomes $$0 = (-1)^{0+0}m_1 (m_1) = m_1^2 .$$ This 
means that $A$ is a cochain complex with differential $d=m_1$ as $m_1$ is a degree 
$1$ map. For $n=2$ we get
$$0 = (-1)^{1}m_2(Id\otimes m_1)+(-1)^{0}m_1 (m_2)+(-1)^{1}m_2 (m_1\otimes Id)$$
which reduces to $m_1 m_2 = m_2(m_1\otimes Id + Id\otimes m_1)$. This means that $m_1$ is 
a derivation with respect to $m_2$ as $m_2$ has degree $2$, usually stated as satisfying 
the Leibniz rule. The standard Leibniz rule comes out of this formula when applying it to 
elements and using the Koszul grading rule.

The third relation tells us that $m_2$ is not necessarily associative, and that the 
associator is given by $m_3$. This is usually referred to as $m_2$ being associative up 
to homotopy, which gives $A_\infty$-algebras their other name, strong homotopy associative 
algebras or sha-algebras for short. If $m_3=0$ (or $m_1=0$) this relation reduces to the 
associator being zero, which means that $m_2$ is an associative product. 

\begin{definition}
    A dg-algebra is an $A_\infty$-algebra $(A, m)$ where $m_i = 0$ for 
    $i \geq 3$. For simplicity we usually just denote it by $A$. 
\end{definition}

By the equations above describing the Stasheff identities, this definition is equivalent 
to the classical definition, i.e. a $\Z$-graded vector space with an associative product 
and a differential satisfying the Leibniz rule. 

\begin{definition}
    Let $(A, m^A)$ and $(B, m^B)$ be $A_\infty$-algebras. A morphism of 
    $A_\infty$-algebras $f:A\longrightarrow B$, also called $A_\infty$-morphism, is a 
    family of linear maps $f_n:A^{\otimes n}\longrightarrow B$ of degree $1-n$, such that
    $$\sum_{n = r+s+t}(-1)^{r+st}f_{r+1+t}(id^{\otimes r}\otimes m_s^A \otimes id^{\otimes t}) = \sum_{k=1}^{n}\sum_{n=i_1+\cdots i_k}(-1)^{u_k} m_k^B(f_{i_1}\otimes \cdots \otimes f_{i_k})$$
    where $u_k=\displaystyle \sum_{t=1}^{k-1}t(i_{k-t}-1)$.     
\end{definition}

We call $f$ an $A_\infty$-isomorphism, or an isomorphism of $A_\infty$-algebras, if $f_1$ 
is an isomorphism of chain complexes.

Since any $A_\infty$-algebra $(A, m)$ has a map $m_1$ such that $m_1^2=0$, we can also 
create its cohomology algebra, denoted $H(A)$. The cohomology algebra of a dg-algebra 
is a graded associative algebra with the induced product from $A$, which we can treat as 
a dg-algebra by letting it have trivial differential.

Let $f:A\longrightarrow B$ be an $A_\infty$-morphism. We call $f$ an 
$A_\infty$-quasi-isomorphism, or a quasi-isomorphism of $A_\infty$-algebras, if $f_1$ is 
a quasi-isomorphism of chain complexes, i.e. it induces an isomorphism on their cohomology 
algebras.

Notice that if $f_j=0$ for $j\geq 2$ and $m^A_i = 0 = m^B_i$ for $i\geq 3$, i.e. $A$ and 
$B$ are dg-algebras, then this definition reduces to the standard quasi-isomorphisms of 
dg-algebras.

\begin{definition}
    Let $A$ be a dg-algebra. We say $A$ is formal if it is 
    $A_\infty$-quasi-isomorphic to a dg-algebra with trivial differential.        
\end{definition}

A formal dg-algebra is isomorphic to $H(A)$, so we can define a dg-algebra to be 
formal if it is $A_\infty$-quasi-isomorphic to its cohomology algebra. We also remark 
that this is not the most classical definition, which uses a zig-zag of 
dg-quasi-isomorphisms instead of an $A_\infty$-quasi-isomorphism. This is equivalent 
to the one we are using here. See \cite{AHO} and \cite{keller} for further details. 

Formal dg-algebras are very nice algebras that as mentioned have their historical 
upbringing in rational homotopy theory. Examples include the Sullivan algebras of 
Kähler manifolds \cite{DGMS}.